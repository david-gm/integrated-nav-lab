\documentclass[11pt,parskip,naustrian,fleqn,a4paper]{scrartcl}
\usepackage[T1]{fontenc}
\usepackage{a4wide}
\usepackage[utf8]{inputenc}
%\usepackage{naustrian}
\usepackage{babel}
\usepackage{graphicx}
\usepackage{epstopdf}
\usepackage{amsmath}
\usepackage{amsfonts}
\usepackage{color}
\usepackage{booktabs}
\usepackage{fancyvrb}
\usepackage{lmodern}
\usepackage{dcolumn}
%for header
\usepackage{fancyhdr}
%for titlepage
\usepackage{tabularx}
\newcolumntype{C}[1]{>{\centering\arraybackslash}p{#1}} % zentrierte Spalten mit Breitenangabe 
\newcolumntype{R}[1]{>{\raggedleft\arraybackslash}p{#1}} % rechtsbündig mit Breitenangabe 
%Jetzt kann man an Stelle des p{BREITE} einfach C{BREITE} für zentrierte, und R{BREITE} für rechtsbündige Textausrichtung verwenden.
%Tabellen
\usepackage{multirow}
%for pdf editing
\usepackage[pdftex,plainpages=false]{hyperref}
\usepackage[final]{pdfpages}

%define new titlestyles
\usepackage{titlesec}

%line break after paragraph
\titleformat{\paragraph}[hang]{\sffamily\bfseries}{}{0pt}{}
\titlespacing*{\paragraph}{0mm}{10pt}{5pt}

%pdf editing:
\hypersetup{%
	colorlinks=false,
  pdfstartview=FitV} % PDF-Viewer benutzt beim Start bestimmte Seitenbreite

\makeindex

%____________________________________
%Makro für Title erstellt
%#1..Fach, #2..LV-Nummer, #3..Semester, #4..Titel, #5..Untertitel, #6..Name1, #7..MatrNr.1, #8..Name2, #9..MatrNr.2
\newcommand{\titelpro}[6]
{

	%\includegraphics[width=0.2\columnwidth]{Pics/LogoTUGraz}
	\clearpage	

\flushright{\today}

\begin{table}[htbp]
  \centering
    \begin{tabularx}{\textwidth}[t]{@{\extracolsep{\fill}}lcp{0.17\textwidth}}
    \midrule[1.5pt]
    \\   
    \multirow{4}{*}{\includegraphics[width=0.2\columnwidth]{Pics/LogoTUGraz}} & \textsc{\huge{#1}} &  \\ & & \\ & #2 & \\ & #3 &  \\  
    \\
    \midrule[1.5pt]
    \\
    \multicolumn{1}{c}{} & \textsc{\Large{#4}} & \multicolumn{1}{c}{} \\
    \multicolumn{1}{c}{} & \large{\bf #5} & \multicolumn{1}{c}{} \\\\\\\\\\
    {} & {\it erstellt von:} & {}\\
    #6
    %\hline
    \end{tabularx}
\end{table}
	\vspace*{\fill}
}
%____________________________________
%Makro für Header erstellt
\newcommand{\headernew}[5]
{
\pagestyle{fancy}
\renewcommand{\footrulewidth}{0.4pt}
\addtolength{\headheight}{\baselineskip}
\addtolength{\headheight}{10.5pt}

\fancyhead{}
\fancyfoot{}
\lhead{#1}
\chead{\textit{#2}\\\vspace{10pt}\textbf{#3}}
\rhead{#4}
\cfoot{\thepage}
\rfoot{\small#5}
}

%____________________________________
%Makroheader
%\headernew{Navigation}{1.Programm}{Kalman Filter}{David Gmeindl}{MatNr.: 0831307}


%define content of header and footer
%\usepackage{scrpage2}
%\cfoot[\pagemark]{pagemark}
%\automark{section}			%with subsection in top left header: \automark[subsection]{section}
%\clearscrheadfoot
%
% \renewcommand{\sectionmark}[1]{\markboth{\sectionmarkformat #1}{}}
%
%\ihead{\rightmark}
%\ohead{\leftmark}
%\lofoot{Integrated Navigation}
%\lefoot{Integrated Navigation}
%
%\cfoot{\pagemark} 
%
%\setheadsepline{0.4pt}
%\setfootsepline{.4pt}

%____________________________________

\newenvironment{tables}
{\vspace*{12pt}\hspace{12pt}\begin{center}}
{\end{center}}

%____________________________________
%Entsprichtsymbol
\newcommand{\entspricht}{\
rel{\widehat{=}}}

% mehr als 3 Tabellen pro page
\renewcommand\floatpagefraction{.9}
\renewcommand\topfraction{.9}
\renewcommand\bottomfraction{.9}
\renewcommand\textfraction{.1}   
\setcounter{totalnumber}{50}
\setcounter{topnumber}{50}
\setcounter{bottomnumber}{50}

%new definition of vectors
\renewcommand{\vec}[1]{\underline{#1}}