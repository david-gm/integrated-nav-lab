% !TeX root = text.tex

\subsection{Initial Alignment}

Im \textit{Initial Alignment} wird aus den statischen Daten {\tt INS\_static.txt} die Euler Winkel für die Drehmatrix $\mathbf{R_b^L}$ berechnet. Gemessen wurden die spezifische Kraft $f^b$ und die Drehwinkel im inertialen \textit{Body Frame} $\omega_{ib}^b$. Aus der spezifischen Kraft $f^b$
\begin{equation}
 	-\vec{f}^b = \vec{R}_L^b \vec{g}^L
\end{equation}
und dem Drehwinkel $\omega_{ib}^b$
\begin{equation}
 	\vec{\omega}_{ib}^b = \vec{R}_L^b \vec{\omega}_{ie}^L
\end{equation}
lassen sich die Hilfsvektoren $c^b$ und $c^L$ bilden:
\begin{align}
 	\vec{c}^L &= \vec{g}^L \times \vec{\omega}_{ie}^L\\
	 \vec{c}^b &= -\vec{f}^b \times \vec{\omega}_{ib}^b
\end{align}
Es gilt:
\begin{equation}
 	\vec{c}^b(t_k) = \vec{R}_L^b(t_0)\vec{c}^L(t_0)
\end{equation}
und damit das Gleichungssystem
\begin{align}
 	\underbrace{\left[ -\vec{f}^b \vec{\omega}_{ib}^b \vec{c}^b \right]_{t_k}}_{\text{gemessen: }\vec{F}^b(t_k)}& = \underbrace{\vec{R}_L^b(t_0)}_{\text{gesucht}} \underbrace{\left[ \vec{g}^L  \vec{\omega}_{ie}^L \vec{c}^L \right]}_{\text{bekannt} \vec{F}^l(t_0)}\\
	\vec{R}_L^b &= \vec{F}^b \left( \vec{F}^L \right)^{-1}
\end{align}

Die drei Euler Winkel $r, p , y$ ergeben sich aus der Drehmatrix $\mathbf{R_L^b}$ berechnen:
\begin{equation}\label{eq:r_lb}
	\vec{R}_L^b =
	\begin{bmatrix}
		\cos p \cos y & \cos p \sin y & -\sin p\\
		r_{21} & r_{22} & \sin r \cos p\\
		r_{31} & r_{32} & \cos r \cos p
	\end{bmatrix}
\end{equation}

\begin{equation}
 	r = \arctan{\frac{r_{23}}{r_{33}}}, \quad y = \arctan{\frac{r_{12}}{R_{11}}},\quad p = \arcsin{-r_{13}}
\end{equation}

Um die Erdrotation $\omega_{ie}^L$ und lokale Normalschwere $\bar{g}^L$ berechnen zu können, benötigt man Modellparameter (siehe dazu Tabelle \ref{tab:constants} auf Seite \pageref{tab:constants}) und die gegebene WGS-84 Koordinaten des IMU Zentrums (siehe Tabelle \ref{tab:imu_centre} auf Seite \pageref{tab:imu_centre}). Danach sind die Formeln für den Erdrotationsvektor im Standpunkt (nur von $\varphi$ abhängig) und die Formel von Somigliana anzuwenden und in die Hilfsvektoren einzusetzen.
\begin{equation}
 	\vec{\omega}_{ie}^L =
		\begin{bmatrix}
			   \omega_E \cos{\varphi}\\
                	   0\\
                	   \omega_E \sin{\varphi}\\
		\end{bmatrix}
\end{equation}
\begin{equation}\label{eq:g_L}
 	\vec{\bar{g}}^L(t_k) = 
		\begin{bmatrix}
			0\\
			0\\
			\gamma(\varphi (t_k), h(t_k))
		\end{bmatrix}
\end{equation}
mit
\begin{equation}
 	\gamma (\varphi, h) \approx \gamma(\varphi) \cdot \left ( 1 -\frac{2}{a} \left ( 1+f+m-2f\sin^2{\varphi}\right) h +\frac{3}{a^2}h^2 \right),
\end{equation}
\begin{equation}
 	f = \frac{a-b}{a} ,\quad m = \frac{\omega_E^2a^b}{GM_E}
\end{equation}
und der Formel von Somigliana:
\begin{equation}\label{eq:somigliana}
 	\gamma (\varphi) = \frac{a \gamma_a \cos^2{\varphi} + b \gamma_b \sin^2{\varphi}}{\sqrt{a^2 \cos^2{\varphi} + b^2 \sin^2{\varphi}}}
\end{equation}

\subsection{Strapdown Algorithm}

Der Strapdown Algorithmus berechnet aus den gemessenen Größen den Positionsvektor $x^e$ und denn Geschwindigkeitsvektor $v_e^L$. Gegeben sind die Messungen der spezifischen Kraft $f^b$ der Akzelerometer sowie die Drehwinkel $\omega_{ib}^b$ der Gyrometer. Zusätzlich müssen für die erste Epoche der inertiale Positionsvektor $x^e(t_0)$ und Geschwindigkeitsvektor $v_e^L(t_0)$ gegeben sein. Der inertiale Positionsvektor ist gegeben (siehe Tabelle \ref{tab:imu_start}, Seite \pageref{tab:imu_start}), der Geschwindigkeitsvektor wird in der ersten Epoche mit $\mathbf{0}$ angenommen. Die initiale Attitude Matrix $\mathbf{R_b^L}(t_0)$ wird aus den Euler Winkeln des \textit{Fine Alignment} berechnet (siehe Tabelle \ref{tab:fine_alignment}, Seite \pageref{tab:fine_alignment}).

Die Messungen des Files {\tt INS\_kinematic.txt} wurden auf 50\;Hz gesampled und danach die Feler aus Tabelle \ref{tab:biases} angebracht. 
\begin{align}
 	\vec{\omega}_{ib}^b &= \vec{\omega}_{ib \text{ meassured}}^b - \Delta \vec{\omega}_{ib}^b\\
    	\vec{f}^b &= \vec{f}^b_{\text{meassured}} - \Delta \vec{f}^b
\end{align}

\paragraph{Algorithmus}

\paragraph{Step 1}

Die Attitude Matrix $\mathbf{R_b^L}$ für die nächste Epoche wird aus der Differentialgleichung berechnet:
\begin{align}
	\vec{\dot{R}}_b^L(t_{k+1}) &= \vec{R}_b^L(t_k)\vec{\Omega}_{ib}^b(t_{k+1}) - \vec{\Omega}_{iL}^L(t_k)\vec{R}_b^L(t_k)\\
	\vec{R}_b^L(t_{k+1}) &= \vec{R}_b^L(t_k) + \Delta t\; \vec{\dot{R}}_b^L(t_{k+1})
\end{align}

\begin{tabularx}{\textwidth}{l c X}
	$\qquad\vec{R}_b^L(t_k)$ & ... & In der ersten Epoche wird $\vec{R}_b^L$ mit den Werten des \textit{Fine Alignment} berechnet. In allen anderen Epochen wird r, p, y der Vorepoche für die Berechnung herangezogen (siehe Formel \ref{eq:r_lb}). Danach wird erst $\vec{R}_b^L(t_{k+1})$ mit $\vec{R}_b^L$ berechnet.\vspace{.5em}\\
	$\qquad\vec{\Omega}_{ib}^b(t_{k+1})$& ... & Messungen des Gyrometers zum Zeitpunkt $t_k$. Die Matrix ist eine schief-symmetrische Matrix gebildet mit dem Vektor $\vec{\omega}_{ib}^b$. \vspace{.5em}\\
	$\qquad\vec{\Omega}_{iL}^L(t_k)$ & ... & $\vec{\omega}_{iL}^L$ wird aus der Position zum Zeitpunkt $t_k$  und den Komponenten des Geschwindigkeitsvektors berechnet. Die Matrix $\vec{\Omega}_{iL}^L(t_k)$ ist eine schief-symmetrische Matrix gebildet aus dem Vektor $\vec{\omega}_{iL}^L$.
\end{tabularx}

Für den Drehwinkel Vektor $\vec{\Omega}_{iL}^L(t_k)$  muss die Geschwindigkeit der IMU bekannt sein:
\begin{equation}
 	\vec{v}_e^L = \begin{bmatrix}v_n\\v_e\\v_d\end{bmatrix}=\begin{bmatrix}(M+h)\dot{\varphi}\\(N+h)\cos{\varphi} \dot{\lambda}\\-\dot{h}\end{bmatrix}
\end{equation}

mit den Krümmungsradien N und M:
\begin{equation}
	M = \frac{c}{V^3}, \quad N = \frac{c}{V}
\end{equation}
\begin{equation}
 	c = \frac{a^2}{b}, \quad V=\sqrt{1+e’^2\cos^2{\varphi}}, \quad e’^2 = \frac{a^2-b^2}{b^2}
\end{equation}

Damit können die Geschwindigkeiten in geographischer Länge und Breite berechnet werden:
\begin{align}
 	\dot{\varphi} & = \frac{v_n}{M+h}\\
	\dot{\lambda} & = \frac{v_e}{(N+h)\cos{\varphi}}
\end{align}

Der Vektor der Winkelgeschwindigkeiten wird mit der Erdrotation berechnet:
\begin{equation}
 	\vec{\omega}_{iL}^L = \vec{\omega}_{ie}^L +\vec{\omega}_{eL}^L = 
	\begin{bmatrix}
		(\dot{\lambda}+\omega_E)\cos{\varphi}\\
		-\dot{\varphi}\\
		-(\dot{\lambda}+\omega_E)\sin{\varphi}\\
	\end{bmatrix}
\end{equation}

\paragraph{Step 2}

Im zweiten Schritt wird die spezifische Kraft vom \textit{Body Frame} in den \textit{Local Level Frame} gedreht.
\begin{equation}
 	\vec{f}^L(t_{k+1}) = \vec{R}_L^b(t_{k+1}) \vec{f}^b(t_{k+1})
\end{equation}

\paragraph{Step 3}

Mit der aktuellen Position $x(\varphi,\lambda,h,t_k)$ wird die Normalschwere für den Zeitpunkt $t_k$ berechnet, welche in den Akzelerometermessungen mitgemessen wird und daher berücksichtigt werden muss. Die Formeln \ref{eq:g_L} - \ref{eq:somigliana} berechnen die Normalschwere.

\paragraph{Step 4}

Auch die Corioliskraft wirkt in den Vektor der spezifischen Kraft mit ein und muss daher berücksichtigt werden. Abhängig ist diese von der aktuellen Position $x(\varphi,\lambda,h,t_k)$, der Erdrotation $\omega_E$ und der Geschwindigkeit $v_e$.
\begin{equation}
	\vec{\omega}_{ie}^L =
		\begin{bmatrix}
			\omega_E \cos \varphi \\ 
 			0 \\ 
			-\omega_E \sin \varphi
		\end{bmatrix}
\end{equation}
\begin{equation}
	\vec{f}_{\text{Coriolis}}^L = - (\vec{\Omega}_{iL}^L+\vec{\Omega}_{ie}^L) \vec{v}_e^L
\end{equation}

\paragraph{Step 5}

Der aktuelle Beschleunigungsvektor resultiert aus spezifischer Kraft $f^L$, der Normalschwere $\bar{g}^L$, und der Corioliskraft $f_{\text{Coriolis}}^L$.
\begin{equation}
	\vec{\dot{v}}_e^L(t_{k+1}) = \vec{f}^L(t_{k+1}) + \vec{\bar{g}}^L(t_k) + \vec{f}_{\text{Coriolis}}^L(t_k)
\end{equation}
Durch Integration erhält man den Geschwindigkeitsvektor für die nächste Epoche $t_{k+1}$.
\begin{equation}
 	\vec{v}_e^L(t_{k+1}) = \vec{v}_e^L(t_k) + \Delta t \; \vec{\dot{v}}_e^L(t_{k+1})  
\end{equation}

\paragraph{Step 6}

Die zweite Navigationsgleichung wird für die Berechnung der Geschwindigkeit im Earth-Fixed System verwendet.
\begin{equation}
 	\vec{\dot{x}}^e(t_{k+1}) = \vec{R}_L^e(t_k) \vec{v}_e^L(t_{k+1})
\end{equation}
Die Drehmatrix $\mathbf{R}_L^e(t_k)$ wird aus der aktuellen Position $x(\varphi,\lambda,h,t_k)$ berechnet:
\begin{equation}
	\vec{n} =	\begin{bmatrix}
			-\sin\varphi \cos\lambda \\ 
			-\sin\varphi \sin\lambda \\ 
			\cos\varphi
		\end{bmatrix}, \quad
	\vec{e} =	\begin{bmatrix}
			-\sin\lambda \\  
			\cos\lambda \\ 
			0
		\end{bmatrix}, \quad
	\vec{d} =	\begin{bmatrix}
			-\cos\varphi \cos\lambda \\ 
			-\cos\varphi \sin\lambda \\ 
			-\cos\varphi
		\end{bmatrix}
\end{equation}
\begin{equation}
 	\vec{R}_L^e(t_k) = \vec{R}_L^e(\varphi,\lambda) = \begin{bmatrix}\vec{n}& \vec{e}& \vec{d}\end{bmatrix}
\end{equation}
 Durch Integration erhält man die aktuelle Position des INS.
\begin{equation}
 	\vec{x}^e(t_{k+1}) = \vec{x}^e(t_k) + \Delta t \; \vec{\dot{x}}^e(t_{k+1})
\end{equation}

\subsection{Kalman Filter}

\subsubsection{Beobachtungsmodell}
\paragraph{Beobachtungsvektor}

\begin{equation}
	\vec{z}_k = \begin{bmatrix}
		x_{GPS}&y_{GPS}&z_{GPS}&x_{INS}&y_{INS}&z_{INS}
	\end{bmatrix}^T
\end{equation}

\paragraph{Beobachtungsmodell für GPS}

Beobachtungsgleichung für die GPS Beobachtungen:
\begin{equation}
 	\vec{z}_{k\;(1)} = \vec{H}_{k\;(1)} \vec{x}_k + (\vec{v}_k) = \begin{bmatrix}x_{GPS}\\y_{GPS}\\z_{GPS}\end{bmatrix}_k = 
		\begin{bmatrix}
			1&0&0&0&0&0\\
			0&1&0&0&0&0\\
			0&0&1&0&0&0
		\end{bmatrix} \vec{x}_k
\end{equation}
Kovarianzmatrix der Beobachtungen:
\begin{equation}
 	\vec{R}_k^{GPS} = \begin{bmatrix}\sigma_x^2&0&0\\&\sigma_y^2&0\\&&\sigma_z^2\end{bmatrix}
\end{equation}

\paragraph{Beobachtungsmodell für INS}

Beobachtungsgleichung für die INS Beobachtungen:
\begin{equation}
 	\vec{z}_{k\;(2)} = \vec{H}_{k\;(2)} \vec{x}_k + (\vec{v}_k) = \begin{bmatrix}x_{INS}\\y_{INS}\\z_{INS}\end{bmatrix}_k = 
		\begin{bmatrix}
			1&0&0&0&0&0\\
			0&1&0&0&0&0\\
			0&0&1&0&0&0
		\end{bmatrix} \vec{x}_k
\end{equation}
Kovarianzmatrix der Beobachtungen:
\begin{equation}
 	\vec{R}_k^{INS} = \begin{bmatrix}\sigma_{x}^2&0&0\\&\sigma_{y}^2&0\\&&\sigma_{z}^2\end{bmatrix}
\end{equation}

\subsubsection{Dynamisches Modell}

\paragraph{Prädiktion mit Transition Matrix}

\begin{equation}
	\vec{\tilde x}_{k+1}  =  \vec{\Phi}_k \vec{\hat x}_k = 
		\begin{bmatrix}
 			\tilde{x}\\\tilde{y}\\\tilde{z}\\\tilde{\dot{x}}\\\tilde{\dot{y}}\\\tilde{\dot{z}}
		\end{bmatrix}_{k+1} = 
		\begin{bmatrix}
			1&0&0&\Delta t&0&0\\
			&1&0&0&\Delta t&0\\
			&&1&0&0&\Delta t\\
			&&&1&0&0\\
			&&&&1&0\\
			&&&&&1
		\end{bmatrix}_k
		\begin{bmatrix}
 			\hat{x}\\\hat{y}\\\hat{z}\\\hat{\dot{x}}\\\hat{\dot{y}}\\\hat{\dot{z}}
		\end{bmatrix}_k
\end{equation}

\paragraph{Systemrauschen}

Die Auswirkung der Störbeschleunigungen $\vec{a}$ auf die Beobachtungen wird mit $\mathbf{N}$ beschrieben.
\begin{equation}
	\begin{bmatrix}\Delta x\\\Delta y\\\Delta z\\\Delta \dot x\\\Delta \dot y\\\Delta \dot z\end{bmatrix} = 
	\begin{bmatrix}
	\frac{1}{2} \Delta t & 0 & 0\\
	0& \frac{1}{2} \Delta t & 0\\
	0& 0 & \frac{1}{2} \Delta t\\
	\Delta t & 0 & 0\\
	0&\Delta t & 0\\
	0& 0 & \Delta t
	\end{bmatrix}
	\begin{bmatrix}
	\ddot x\\
	\ddot y\\
	\ddot z\\
	\end{bmatrix} = \vec{N} \;\vec{a}
\end{equation}

Kovarianzmatrix der Systemrauschvariablen (der Störbeschleunigungen):
\begin{equation}
 	\vec{S} = \vec{S}_{\vec{a}} = \begin{bmatrix}\sigma_{\ddot x}^2 & 0 & 0\\&\sigma_{\ddot y}^2 & 0\\ &&\sigma_{\ddot x}^2\end{bmatrix}
\end{equation}

Kovarianzmatrix des Systemrauschens:
\begin{equation}
 	\vec{Q}_k = \vec{N}\;\vec{S}\;\vec{N}^T
\end{equation}

\subsubsection{Filter}

1.Schritt: Time Update (Prädektion)
\begin{eqnarray}
 	\vec{\tilde{x}}_{k+1} & = & \vec{\Phi}_k \vec{\hat x}_k\\
	\vec{\tilde P}_{k+1} & = & \vec{\Phi}_k \vec{P}_k \vec{\Phi}_k^T + \vec{Q}_k
\end{eqnarray}

2.Schritt: Berechnung der Kalman Gewichtsmatrix
\begin{eqnarray}
	\vec{K}_k = \vec{\tilde P}_k \vec{H}_k^T (\vec{H}_k \vec{\tilde P}_k \vec{H}_k^T + \vec{R}_k)^{-1}
\end{eqnarray}

3.Schritt: Beobachtungsupdate (Korrekturschritt)
\begin{eqnarray}
	\vec{\hat x}_k & = & \vec{\tilde x}_k + \vec{K}_k (\vec{z}_k - \vec{H}_k \vec{\tilde x}_k)\\
	\vec{P}_k & = & (\mathbf{I}-\vec{K}_k \vec{H}_k) \vec{\tilde P}_k
\end{eqnarray}

\subsubsection{Messlücken}

Im Fall von Messlücken in den GPS Daten wird das Design nur mehrauf die Komponenteix der Beobachtungen der INS reduziert. Es werden nur noch die Designmatrix $\vec{H}_{k(2)}$ und die Kovarianzmatrix der Beobachtungen $\vec{R}_k^{INS}$ der INS verwendet.

\subsubsection{Typen von Kalmanfiltern}
Um die Auswirkungen von verschiedenen Kalmanfilter-Typen zu untersuchen, wurden der ungekoppelte und der lose gekoppelte Kalman-Filter implementiert.

\paragraph{Ungekoppelter Kalmanfilter}

Der ungekoppelte Kalmanfilter erhält vorprozessierte Daten von den GPS Daten und dem Strapdown Algorithmus. GPS Daten und die Koordinaten des Strapdown Algorithmus sind im erdfesten Referenzsystem gegeben.
Im ungekoppelten Kalmanfilter werden aus beiden Inputs die Koordinaten der Position berechnet. Es erfolgt kein Feedback an ein System.

\paragraph{Lose gekoppelter Kalmanfilter}

Der lose gekoppelte Kalmanfilter gibt als Feedback die Position $\vec{x}_e$ der aktuellen Epoche an den Strapdown Algorithmus für die nächste Epoche zurück. Da die GPS Koordinaten mit einer Samplerate von 1~Hz und die Koordinaten der INS mit 50~Hz gegeben sind, muss der Kalmanfilter in den Strapdown Algorithmus implementiert werden. Somit fließt jedes Ergebnis des Strapdown Algorithmus in den Kalmanfilter und nur jede 50. GPS-Koordinate in den Kalmanfilter ein.